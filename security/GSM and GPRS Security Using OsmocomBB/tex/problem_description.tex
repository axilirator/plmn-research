\begin{titlingpage}

\noindent
\begin{tabular}{@{}p{4cm}l}
\textbf{Title:} 	& \thetitle \\
\textbf{Student:}	& \theauthor \\
\end{tabular}

\vspace{4ex}
\noindent\textbf{Problem description:}
\vspace{2ex}

\noindent 

The OsmocomBB project~\cite{osmocombb_2015} aims to create a free
and open source GSM baseband software implementation, which enhances
cheap and accessible compatible phones by giving access to their inner
workings. Thus, it can be used to analyze GSM and GPRS security
functionalities.

There are four goals to this thesis and the first one is to set up and
understand the OsmocomBB software, and to use it to acquire a solid
practical knowledge of GSM and GPRS with a focus on the security
aspects. The second one is to reproduce and understand the feasibility
and efficiency of a passive attack~\cite{munaut_wideband_2010} which
uses a modified version of OsmocomBB along with cheap compatible mobile
phones to eavesdrop on GSM. The third goal is to do the same with
another passive attack~\cite{melette_gprs_2011} which allows to
eavesdrop on GPRS using almost the same set of tools. Finally, the last
goal is to analyze the security configuration of mobile networks at
various locations and to check whether mobile operators implemented
solutions to prevent these attacks.

\vspace{6ex}

\noindent
\begin{tabular}{@{}p{4cm}l}
\textbf{Responsible professor:} 	& \theprofessor \\
\textbf{Supervisor at ULB:}			& \thesupervisor \\
\end{tabular}

\end{titlingpage}
