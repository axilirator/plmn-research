\chapter{Related projects}
\label{chap:related_projects}

    This chapter gives some context around the \proj{OsmocomBB} project
    and provides references to selected works on the same topic. It
    shows that research on \gls{gsm} security have been carried out for
    a long time, and highlights its milestones.

    \section{Nokia DCT3}

      For a few years, \comp{Nokia} shipped its DCT3 family of mobile
      phones with a remote logging facility used for debugging. This
      allowed various projects to exist because it gave out a lot of
      otherwise hidden information to hackers and independent
      researchers.

      One of the project exploiting this situation was \proj{Project
      Blacksphere} around 2003~\cite{project_????}. This project led to
      the creation of a series of tools which helped to debug \gls{gsm}
      and created a community around it. One of these tools is
      \prog{Debug Tracing}, which is provided in the \proj{Gammu}
      software when using the \code{nokiadebug} argument. It provides
      very useful debug traces that can also help understand how the
      network works. For example, it is possible to receive most of the
      layer 2 and 3 messages that the phone processes. \name{Glendrange}
      et al.\ showed examples of results obtained using this
      software~\cite[p.~89]{glendrange_decoding_2010}.

      \prog{NetMonitor} is a hidden software available on some phones
      that allows it to display various network and phone
      parameters~\cite{wiacek_netmonitor_2002}. It is also possible to
      enable this software on DCT3 phones using \proj{Gammu} with the
      argument \code{nokianetmonitor}. This gives information about the
      phone and network parameters. Again, \name{Glendrange} et al.\
      showed examples of results found with \prog{NetMonitor}
      ~\cite[p.~85]{glendrange_decoding_2010}.

      \prog{Debug Tracing} was also used to reverse engineer the DCT3
      phones and to provide flashing, upgrading, or modding
      capabilities. This led to projects like \proj{MADos}: an open
      source alternative firmware for DCT3 phones created by
      \name{g3gg0} and \name{krisha}~\cite{project_????}. It can
      probably be considered as the ancestor of \proj{OsmocomBB}. A big
      community existed around this project because it allowed people to
      install custom applications on the phones, new games for example.
      The source code of \proj{MADos} is still available after more than
      ten years~\cite{index_????}.

      \prog{NetMonitor} and \prog{Debug Tracing} were both used to
      gather information about the network. This seemed to be very
      useful in the beginning of \gls{gsm} security research and helped
      to create a community around it, while paving the way for more
      ambitious projects.

    \section{THC projects}

      \gls{thc} is a group of hackers which was active in the \gls{gsm}
      development community starting from the beginning of 2007. They
      introduced the \proj{GSM Software Project} and the \proj{A5
      Cracking Project} at the CCCamp in 2007~\cite{hulton_a5_2007}. The
      \proj{GSM Software Project}, also called \proj{GSM Scanner or
      Sniffer Project}, led to the creation of various tools. Its goal
      was to analyze and understand the \gls{gsm} network and to build a
      \gls{gsm} receiver for less than \SI{1000}{\$}. After exploring
      various ideas, the development focused on \proj{GNU Radio} and
      \gls{usrp} devices~\cite{gsm_2009}.

      \iffalse
      \fxnote{http://docslide.us/download/document/?id=0IRArNhE%2FvyDjqnteKTcr0WDHRsWWqgrwIyHSob8UYyheLQr8aBjMcDYQ9j0IRBD35jgsCv%2FXmMYnc0ZIjWboA%3D%3D}
        \fxnote{https://content.uploady.com/download/Cracking-a5-THC-Wiki.pdf?f=Cracking-a5-THC-Wiki.pdf&fid=3tZsWm3Zvm_&p=A9URDey47wx&m=application%2Fpdf&s=1&u=https%3A%2F%2Fwww.uploady.com%2Fdownload%2F3tZsWm3Zvm_%2FBcbEAesbc8a3bYmO&tp=remote&t=1430755990&ex=172800&ip=2180108371&h=7f088c6c4f36fdfb3b0118f980480dcbcc6d4f17}
      \fi

      The main tools that were released were \prog{gssm}, \prog{gsmsp},
      \prog{gsm-tvoid}, and \prog{gsmdecode}. The three first tools were
      used along a \gls{usrp} and \proj{GNU Radio} to capture a limited
      subset of unencrypted traffic, and to demodulate and decode its
      layer 1 to create layer 2 packets. According to \name{Harald
      Welte}, \prog{gssm} and \prog{gsmsp} were two early
      implementations by \name{Joshua Lackey} which stayed at the alpha
      level~\cite{welte_airprobe:_2009}. \prog{Gsm-tvoid} was developed
      by someone under the pseudonym of \name{Tempest Void} and stayed
      the best decoder for a long time, even including a \gls{gui}.
      \prog{Gsmdecode} does not have the same purpose and is used to
      decode the layer 2 \gls{gsm} messages from the DCT3 phones or from
      \prog{gsm-tvoid}. It converts hexadecimal bytes to human readable
      data and is similar to \proj{Wireshark} from that point of view.

      Another project from \gls{thc} is the \proj{TSM30 project}, which
      aimed to modify the firmware of the \comp{Vitelcom} TSM30 mobile
      phone to receive and send arbitrary traffic. It is apparently
      based on an older Spanish project called \proj{TuxSM}, but more
      information was hard to find~\cite{roura_proyecto_2005}. The TSM30
      phone uses a Calypso based platform, which is the targeted
      platform for the \proj{OsmocomBB} project. This was preferred over
      the Nokia DCT3 platform because the source code of the TSM30
      firmware and some documentation were leaked, making the work
      easier~\cite{opentsm_2009,texas_instruments_hercrom400g2_2000-1,texas_instruments_hercrom400g2_2000,sokolov_sharing_2011}.
      The project eventually stopped, maybe because the phones were hard
      to find.

    \section{Attacks on A5}

      \gls{gsm} cryptography is based on a set of algorithm called the
      A5 cipher family. These algorithms have never been released
      officially to the public, but were partially leaked as soon as
      1994 when \name{Ross Anderson} received some incomplete
      documentation \textquote{anonymously in two brown
      envelopes}~\cite{anderson_a5_1994}. The A5/1 and A5/2 algorithms
      were completely reversed engineered by \name{Marc Briceno},
      \name{Ian Goldberg}, and \name{David Wagner} in 1999, and a
      pedagogical implementation was
      proposed~\cite{briceno_pedagogical_1999}.

      Several attacks were published throughout the years. The first
      analysis of A5 was published by \name{Jovan Dj. Golić} in
      1997~\cite{golic_cryptanalysis_1997}. In 2003, \name{Elad Barkan},
      \name{Eli Biham}, and \name{Nathan Keller} demonstrated a
      practical attack breaking A5/2 in less than a second using a
      ciphertext-only attack requiring a few dozen milliseconds of
      encrypted data~\cite{barkan_instant_2003}. Other practical attacks
      on A5/1 were attempted, but none of them resulted in an effective
      and public way to break A5/1 as it is implemented in \gls{gsm}.

      The \proj{A5 Cracking Project} emerged from \gls{thc} around 2007,
      as stated in the previous section. This project aimed to develop a
      practical way of cracking A5/1 by reducing the time and the price
      of the attack. To do so, they focused on known plaintext since it
      is common in \gls{gsm}, and decided to apply a time memory
      trade-off by building rainbow tables using \glspl{fpga}. They
      claimed to obtain very good results compared to the previous
      methods, because of their use of rainbow
      tables~\cite{hulton_intercepting_2008}. These tables were supposed
      to be released around the second quarter of 2008 but were not, and
      the project seems to be abandoned now.

    \section{Berlin A5/1 rainbow table set and \proj{Kraken}}
    \label{sec:berlin}

      The main problem with the projects described in the previous
      section were the centralized development and computation, which
      created a single point of failure. In August 2009, \name{Sascha
      Krißler} and \name{Karsten Nohl} introduced a new project at
      Hacking at Random~\cite{nohl_subverting_2009}.
      
      This project was different than the previous ones because it tried
      to allow anyone to work on the computation of the tables and share
      them, thus distributing the responsibility and diminishing the
      possibility of failure. The programs used to compute the table
      were optimized during the whole life of the project. What was
      supposed to take three months on 80 GPUs~\cite{nohl_gsm:_2009}
      finally took four weeks on four GPUs~\cite{nohl_breaking_2010}.
      These programs were available on the project website which is now
      offline, but they can be downloaded along with
      \proj{Kraken}~\cite{kraken_????}. This subject is discussed in
      further details on the project wiki, and by \name{Glendrange} et
      al.~\cite{glendrange_decoding_2010,srlabs_wiki_????}. The torrent
      files can be found on the project wiki as well, and tables can be
      bought from people willing to sell it.

      On the 16th of June 2010, \name{Frank A. Stevenson} publicly
      announced the completion of the set~\cite{stevenson_[a51]_2010-1}.
      These tables are often called the Berlin A5/1 rainbow table set
      and seem to be the first which were publicly available. This does
      not mean that A5/1 was crackable in practice then. Indeed, a tool
      was needed to compute the session key from some keystream using
      the tables, and this is \proj{Kraken}. \proj{Kraken} is the first
      public A5/1 cracker, created by \name{Frank A. Stevenson}, and
      which uses the Berlin table set. It was released on the 16th of
      July 2010~\cite{stevenson_[a51]_2010}. This project was a leap
      forward in the demonstration of \gls{gsm} insecurity: from some
      keystream, it found a session key in a matter of seconds on a
      mainstream computer.
      
    \section{\proj{Airprobe}}
    \label{sec:airprobe}

      \proj{Airprobe} is a follow up to the \gls{thc} \proj{GSM Software
      Project}. It was introduced at Hacking at Random in August 2009 by
      \name{Harald Welte}~\cite{welte_airprobe:_2009,airprobe_????}.
      This project is used to capture, demodulate and decode \gls{gsm}
      traffic using \gls{usrp} devices and \proj{GNU Radio}. The goal
      was again to raise awareness about \gls{gsm} security.

      \proj{Airprobe} introduced a new tool to the \gls{thc} suite:
      \prog{gsm-receiver}. It was written by \name{Piotr Krysik} and,
      according to \name{Harald Welte}, it has a much better decoding
      accuracy than the other ones. \name{Glendrange} et al.\ showed how
      to intercept \gls{gsm} traffic using \prog{gsm-tvoid} or
      \prog{gsm-receiver}, along with \prog{gsmdecode} or
      \proj{Wireshark}~\cite[p.~111]{glendrange_decoding_2010}. This
      example only works on unencrypted traffic.

      After the Berlin tables set and \proj{Kraken} were released, it
      became possible to do the same for encrypted traffic as well. This
      was shown by \name{Karsten Nohl} at Black Hat 2010 one week after
      the release of
      \proj{Kraken}~\cite{nohl_breaking_2010,srlabs_decrypting_????}.
      \proj{GNU Radio} was used to record data from the air,
      \proj{Airprobe} to parse the control data and extract the
      keystream, \proj{Kraken} to find the A5/1 session key, and
      \proj{Airprobe} again to decode the decrypted voice. To find some
      keystream, known plaintext has to be found as well. This is
      possible since a lot of messages in \gls{gsm} are predictable, as
      was already shown by \name{Karsten Nohl} and \name{Chris Paget} in
      2009~\cite[p.~19]{nohl_gsm:_2009}.

      Despite all these progresses, the \proj{Airprobe} community faced
      several problems which were difficult to solve. Even
      though some work has been done on these two problems, they were
      solved by using \proj{OsmocomBB}, which takes a different approach
      and provides a better signal quality.

    \section{OsmocomBB}

      The development for \proj{OsmocomBB} started in January
      2010~\cite{welte_announcing_2010}, and a demonstration was given
      by Harald Welte at the end of the year at
      Hashdays~\cite{welte_osmocombb:_2010}. This project aims to
      provide a \gls{foss} implementation of a mobile phone \gls{gsm}
      baseband chipset. The goal was once again to better understand
      \gls{gsm} and raise awareness of its security issues. Apparently,
      only two other projects tried something similar: MADos and the
      \gls{thc} \proj{TSM30} project, both described in the previous
      sections. \proj{OsmocomBB} allows researchers to control the
      baseband chipset of a mobile phone. This makes it easier to
      analyze the received traffic, to send arbitrary data, and so on.
      Various applications are available to do so, and it is possible to
      modify them and create new ones. This will be demonstrated in this
      thesis.

