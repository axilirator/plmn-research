\chapter{Introduction} \label{chap:introduction}

%starting points, objective, results, conclusion
  
%http://www.theregister.co.uk/2012/11/09/20_years_of_gsm_digital_phones/
  %http://www.euronews.com/2015/06/01/belgium-hanging-up-on-the-phone-booth/
  %http://www.fiercewireless.com/europe/story/frances-operators-sign-accord-cover-all-mobile-not-spots-2020/2015-05-22?utm_medium=rss&utm_source=rss&utm_campaign=rss
  %http://www.fiercewireless.com/story/facebook-launches-facebook-lite-app-designed-2g-networks-emerging-markets/2015-01-27
  %http://indianexpress.com/article/technology/social/exclusive-skype-to-roll-out-made-for-india-app/
  %http://www.fiercewireless.com/europe/story/telenor-norway-will-ditch-3g-2g-lte-rollouts-gather-pace-northern-territori/2015-06-03?utm_medium=rss&utm_source=rss&utm_campaign=rss

  More than 20 years after its introduction, the \gls{gsm} is still
  relevant as a cheap and effective legacy system~\cite{cox_20_2012}.
  While phone booths are gradually removed from the streets, \gls{gsm}
  is now considered by some countries as a backup communication system
  which should be available everywhere, and might outlive
  3G~\cite{belgium:_2015,morris_frances_2015,carroll_telenor_2015}.
  Moreover, \gls{gprs} is still widely used in developing countries:
  \comp{Skype} or \comp{Facebook} recently provided specifically
  designed versions of their mobile applications aimed at the 2G data
  rate~\cite{goldstein_facebook_2015,swami_exclusive:_2015}.

  Yet, 20 years is a really long time in technology, and it shows. When
  \gls{gsm} networks were first deployed, the equipment was so expensive
  that it could only be acquired by phone operators. Now that cheap
  equipment and \gls{foss} projects are available, a wide attack surface
  is accessible. One of this project is \proj{OsmocomBB}, which provides
  a \gls{foss} implementation of a \gls{gsm} protocol stack. The
  motivation behind this thesis is to explore the impact of this project
  on the security of \gls{gsm} and \gls{gprs}. 

  \section{Problem description evolution}
  \label{sec:pb}

  The problem description defined originally and included at the
  beginning of this thesis had to be adapted along the way. This was
  done to follow the evolution of the work and the better understanding
  of the systems and their capabilities. The objective of this thesis
  was: exploring the available eavesdropping attacks on \gls{gsm} and
  \gls{gprs} using \proj{OsmocomBB}. This objective was modified in two
  ways. Firstly, the focus was switched to mostly consider \gls{gsm} at
  the expense of \gls{gprs}. Secondly, a new set of attacks was
  considered.

  The focus of this thesis on the security of \gls{gprs} was reduced for
  several reasons. Firstly, \gls{gprs} is not supported by
  \proj{OsmocomBB} yet, which makes attacks on this network very hard to
  implement. Secondly, the eavesdropping attack on \gls{gprs} has
  several major limitations, and is thus less interesting than first
  expected. Finally, the \gls{gprs} network is almost completely
  separated from the \gls{gsm} network, and a lot of work would have
  been needed to cover it too. Therefore, the emphasis is clearly on
  \gls{gsm}, but the impact on \gls{gprs} is considered as well. The
  focus also shifted from the eavesdropping attacks to a wider range of
  attacks that could be implemented with \proj{OsmocomBB}. A complete
  chapter is dedicated to \gls{dos} attacks, which appeared to be very
  relevant to the topic.

  The four goals originally defined are mostly accomplished. A solid
  practical knowledge had to be developed in order to implement the
  various attacks demonstrated in this thesis. The eavesdropping attack
  on \gls{gsm} was described in details and some steps were implemented,
  but it could not be executed completely. The only public demonstration
  of this attack was performed by \name{Sylvain Munaut}, and no
  implementation or detailed description were publicly available, which
  made this task harder. The eavesdropping attack on \gls{gprs} is a
  variation of the previous one and is therefore described as well, but
  was not tested. Finally, the feasibility of the attacks on Norwegian
  networks was investigated in details, and extended to \gls{dos}
  attacks. The organization of this thesis is described in the next
  section.

  \section{Structure}

  %Report this in chapter introductions.
    This thesis contains eight chapters. \Cref{chap:introduction}, the
    chapter you are reading, introduces the problem, the structure, and
    the methodology of the thesis. \Cref{chap:related_projects} gives
    references to the relevant researches in the field, and provides
    some context around \proj{OsmocomBB} by presenting important related
    projects.

    \Cref{chap:network_architecture} is meant to outline the \gls{gsm}
    and \gls{gprs} networks architectures, and to provide background for
    the following chapters. \Cref{chap:protocol_stack_implementation}
    gives an overview of the protocol stack running on the mobile
    phones, and provides links between the \proj{OsmocomBB} source code
    and the \gls{3gpp} specifications.

    After two theoretical chapters, \Cref{chap:eavesdropping_attacks} is
    mostly dedicated to an eavesdropping attack on \gls{gsm}, but also
    discusses an attack on \gls{gprs}. It details the various steps of
    the attacks and provides an implementation using \proj{OsmocomBB}
    for some of them. \Cref{chap:dos_attacks} is dedicated to \gls{dos}
    attacks using \proj{OsmocomBB}. It provides a description of the
    attacks as well as an implementation for most of them. They are
    aimed at the \gls{gsm} network, but can indirectly impact the
    \gls{gprs} network as well.

    \Cref{chap:security_configuration_of_norwegian_operators} assesses
    the feasibility of these two categories of attacks on Norwegian
    networks. It offers a detailed investigation using the
    implementations introduced in the previous chapters. Finally,
    \Cref{chap:conclusion} is the conclusion. It summarizes the thesis,
    gets back to its limitations and suggests possible future work.
    Appendices are available at the very end of this document, and
    contain a tutorial on the installation and usage of
    \proj{OsmocomBB}, the patches adding the proposed implementations,
    and a small case study on IMSI-catchers detection.

  \section{Methodology}

    The researches presented in this thesis are based on software.
    Therefore, the applied methodology is based on classical software
    engineering procedures. It is composed of three main phases:
    research, development, and
    validation~\cite{sommerville_software_2007}. These phases can be
    mapped on the chapters to some extent.

    The research phase consisted first in establishing the context
    around the project by getting familiar with related works, as
    described in \Cref{chap:related_projects}. Then, it was crucial to
    understand and analyze the \gls{gsm} and \gls{gprs} networks, but
    also the \proj{OsmocomBB} source code. It implied reading and
    understanding the specifications, and linking them with the
    architecture of the program. This is mostly represented in
    \Cref{chap:network_architecture} and
    \Cref{chap:protocol_stack_implementation}. This phase is also
    extended to \Cref{chap:eavesdropping_attacks} and
    \Cref{chap:dos_attacks}, where the attacks had to be understood and
    analyzed.

    The development phase is described in
    \Cref{chap:eavesdropping_attacks} and \Cref{chap:dos_attacks} as
    well, where the programs implementing several steps of the
    eavesdropping attacks and various \gls{dos} attacks are introduced.
    The patches providing these implementations are available in the
    appendices. Finally, the validation phase is included in
    \Cref{chap:security_configuration_of_norwegian_operators}, where
    these implementations were tested and validated on Norwegian
    networks.
