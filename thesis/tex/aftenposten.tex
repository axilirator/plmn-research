\chapter{Aftenposten case study}

    %Show OsmocomBB can be used.
    At the end of 2014, the \comp{Aftenposten}, one of Norway's largest
    newspaper, wrote an article on the presence of \gls{imsi}-catchers
    in the center of Oslo~\cite{timberg_for_2014}. This claim, along
    with the data on which it is based, is investigated in more details
    by Torjus Retterstøl~\cite{torjus}. An interesting note in
    regard to this thesis is the way this data was collected: the
    journalists used very expensive equipment.

    The point of this section is to show how \proj{OsmocomBB} can be
    used in a practical case by researchers, and how the same data can
    be recovered using a slightly modified version of the
    \prog{cell\_log} application available with the \proj{OsmocomBB}
    project.

      This application works as follows. On startup, it scans either a
      restricted range of \gls{arfcn}s or the complete set. For each
      \gls{arfcn}, it makes a series of power level measurements given
      in dBm. This is shown in the following figure.

      \begin{lstlisting}[language=C, numbers=left,
      basicstyle=\footnotesize, breaklines=true, frame=single]
ARFCN 0 -109 -105 -107 -105 -98 -86 -101 -104 -91 -97 -100 -94
ARFCN 12 -105 -98 -97 -107 -101 -104 -107 -109 -99 -87 -101 -103
ARFCN 24 -95 -103 -102 -104 -103 -106 -107 -100 -100 -101 -105 -107
ARFCN 36 -93 -101 -95 -104 -103 -85 -100 -106 -104 -107 -107 -105
ARFCN 48 -108 -108 -105 -95 -100 -101 -93 -100 -102 -95 -102 -87
[...]
ARFCN 967 -106 -107 -99 -93 -104 -106 -91 -104 -104 -106 -109 -106
ARFCN 979 -106 -103 -108 -106 -107 -108 -105 -106 -104 -104 -104 -105
ARFCN 991 -105 -104 -108 -103 -108 -97 -109 -107 -106 -107 -107 -107
ARFCN 1003 -101 -107 -108 -106 -93 -108 -108 -107 -109 -108 -109 -107
ARFCN 1015 -109 -107 -107 -106 -105 -107 -108 -110 -106
      \end{lstlisting}

      When this is done, it tries to synchronize with all the available
      cells, starting with the one with the strongest signal. Then, it
      listens to the broadcast channel until it gets the SI from 1 to 4.
      After that, it sends a Channel Request message so as to get an
      Immediate Assignment message containing the Timing Advance value.
      Finally, it writes the received values in a log file. The output
      is shown in the following figure. It had to be modified to fit the
      page.\\

      \begin{lstlisting}[language=C, numbers=left,
        basicstyle=\footnotesize,
     frame=single, keepspaces=true]
[sysinfo] Tue May 19 16:05:54 2015
ARFCN |MCC          |MNC          |LAC    |cell ID|BSIC |
------+-------------+-------------+-------+-------+-----+
124   |242    Norway| 01   Telenor|0x307b |0x4112 |0,3  |

rx-lev|min-db |max-pwr|C1  |C2  |T3212 |TA |
------+-------+-------+----+----+------+---+
-71   |-110   |   5   |  16|  16|14400 | 1 |

SI2 (neigh.) BA=0: 51,52,53,54,56,57,59,60,61,62,64,65,66,67,68,76,81,
                 122,123,124

SI1 55 06 19 08 40 00 00 20 00 00 00 00 00 00 00 00 00 00 00 79 00 00 2b
SI2 59 06 1a 0e 00 00 00 00 01 08 0f bd bc 00 00 00 00 00 00 ff 79 00 00
SI3 49 06 1b 41 12 42 f2 10 30 7b c8 03 28 54 65 40 79 00 00 80 00 a0 43
SI4 31 06 1c 42 f2 10 30 7b 65 40 79 00 00 80 00 b2 2b 2b 2b 2b 2b 2b 2b
      \end{lstlisting}

    This shows how it is possible to extract the information easily with
    a very cheap phone. Moreover, some useful data is added here
    compared to the data acquired by \comp{Aftenposten}: the value of
    the t3212 timer, which is the time between periodic updates, the
    timing advance, giving information about the distance from the cell, and
    the list of neighbors advertised in the SI2. The SI messages are
    also displayed directly as they appear on the layer 2. The main
    advantage of using \proj{OsmocomBB} is its flexibility: a lot of
    other information could be displayed.\fxnote{ The GPS data is
    missing for now!}

    One of the main difference between this demonstration and the system
    used by Aftenposten is the automatic detection of IMSI-catchers, and
    the alarms that are available. This kind of system is also possible
    using data collected from \proj{OsmocomBB} of course. An
    IMSI-catcher detector was actually developed by \comp{SRLabs} based
    on \proj{OsmocomBB}:
    CatcherCatcher~\footnote{https://opensource.srlabs.de/projects/mobile-network-assessment-tools/wiki/CatcherCatcher}.
    It provides automatic detections and alarms based on various
    criterion.

\newpage

    \section{Patch}

      This patch modifies the output of the \prog{cell\_log}
      application. It was developed on the
      \code{fc20a37cb375dac11f45b78a446237c70f00841c} commit of the
      master branch, and can also be found online:
      \url{https://gitlab.com/francoip/thesis/raw/public/patch/aftenposten.patch}. \\

\lstinputlisting[language=C, numbers=left, basicstyle=\small,
breaklines=true, frame=single]{../misc/aftenposten.patch}
